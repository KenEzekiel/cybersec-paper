
\section{Introduction}

In the rapidly evolving landscape of digital technology, ensuring secure access to systems and sensitive data has become a critical concern. Traditional authentication mechanisms, such as passwords and PINs have long been the cornerstone of securing digital assets. However, the limitation of these methods, ranging from susceptibility to breaches, social engineering attacks, and user mismanagement, highlight the pressing need for more robust and user-friendly solutions. To address these challenges, multifactor authentication (MFA) has emerged as a popular approach, leveraging multiple layers of verification to strengthen security.

Behavioral authentication represents a novel dimension in the field of MFA by analyzing an individual's unique and often subconscious actions to verify identity. Behavioral factors are inherently linked to the individual and difficult to replicate. This makes behavioral authentication particularly resilient to impersonation and credential theft. However, a major flaw to behavioral authentication is that it is highly prone to false detections.

This paper explores the potential of multifactor behavioral authentication (MFBA) which integrates multiple behavioral traits into a unified framework for identity verification. By combining diverse behavioral signals, MFBA enhances the reliability and robustness of authentication systems, reducing the risk of unauthorized access while maintaining user convenience.
