
\section{Introduction}

In the rapidly evolving landscape of digital technology, ensuring secure access to systems and sensitive data has become a critical concern.
User level attacks such as masquerade attacks, intruder can get unauthorized access to the system by exploiting user rights.
One of the key factor to this is the vulnerability of the authentication system
which increase the likelihood of impersonation of legitimate users.
Consequently, this leads to more reliant on secure authentication mechanisms for securing digital assets.  
Traditional authentication mechanisms, such as passwords and PINs have long been the cornerstone of securing digital assets. However, the limitation of these methods, ranging from susceptibility to breaches, social engineering attacks, and user mismanagement, highlight the pressing need for more robust and user-friendly solutions. To address these challenges, multi-factor authentication (MFA) has emerged as a popular approach, leveraging multiple layers of verification to strengthen security.

Behavioral authentication represents a novel dimension in the field of MFA by analyzing an individual's unique and often subconscious actions to verify identity.
Behavioral factors are inherently linked to the individual and difficult to replicate.
These factors include keystroke dynamics, mouse movements, touch gestures, and other biometric traits that can be captured and analyzed to create a unique user profile.
This makes behavioral authentication particularly resilient to impersonation and credential theft.
But behavioral authentication does not serve as the replacement for traditional authentication methods, rather it serves as an additional layer of security. 
However, a major flaw to behavioral authentication is that it is highly prone to false detections.
This is due to the fact that the behavioral data is highly variable and can be affected by external factors.

To address these challenges of behavioral authentication, 
This paper explores the concept of a multi-factor behavioral authentication (MFBA) framework that integrates multiple behavioral traits into a unified system for identity verification.
The key idea is to make use of different behavioral traits to mitigate the limitations of individual modalities.
By combining diverse behavioral signals, MFBA enhances the reliability and robustness of authentication systems, reducing the risk of unauthorized access while maintaining user convenience.

The rest of the paper is structured as follows.
Section II presents the literature review and related papers on behavioral authentication.
Section III analyzes the problem of behavioral authentication and potential problem that may also appear when implementing an MFBA system. 
Section IV presents the proposed solution for a multi-factor behavioral authentication framework.
Section V presents the conclusion and future work.
