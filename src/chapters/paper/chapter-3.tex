\section{Problem Analysis}
\label{sec:problem-analysis}

Behavioral authentication, while promising as a user-friendly and secure alternative to traditional methods, is not without its challenges. This section analyzes the key issues associated with behavioral authentication systems.

\subsection{Accuracy and Reliability}
As outlined in Section \ref{sec:literature-review}, behavioral analysis have relatively less accurate detections when compared to traditional methods such as passwords and PINs. Furthermore, Behavioral data is often noisier than traditional credential-based inputs. Environmental factors, such as changes in device sensitivity, network latency, or external distrubances, can further exacerbate detection innacuracies.

For example as shown in \ref{subsec:typing-style-using-keystrokes-data}, 

There is many factor that come into play 
% Would be nice to throw in some data from chapter-2

\subsection{Dynamic Nature of Behavioral Data}
Unlike static credentials, behavioral traits are non-deterministic, which increases the complexity of accurate classification. The variability in human behavior introduces a significant challenge in achieving consistent results. For instance, typing dynamics may differ based on user stress, fatigue, or minor distractions. This variability can compromise the reliability of authentication systems, especially in high-security environments where precision is critical. False positives and false negatives are particularly problematic in behavioral systems.
% Would be nice to throw in some cases from chapter-2

\subsection{Usability}
While behavioral authentication is designed to be seamless and non-intrusive, its implementation can inadvertently introduce usability issues. False rejections caused by behavioral variability may frustate legitimate users. On the other hand, frequent prompts for secondary authentication mechanisms to compensate for behavioral mismatches can disrupt the user experience and reduce overall satisfaction.
% Would be nice to throw in some cases from chapter-2
