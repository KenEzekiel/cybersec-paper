\section{Literature Review}
\label{sec:literature-review}

Many papers have been published to analyze various methods for behavioral authentication. This section reviews the existing literature on behavioral authentication techniques, their accuracy, their associated challenges, and their potential when integrated into a multifactor framework.

\subsection{Mouse Movement Feature Using Coordinates and Speed}
\label{subsec:mouse-movement-feature-using-coordinates-and-speed}

\parencite{21_Mouse_Coordinate_Speed} This study proposed a behavioral biometric system using random mouse movements, consisting of three modules: data capture, feature extraction, and classification. Participants interacted with a GUI that randomized button locations, and the system recorded mouse coordinates and timestamps to extract features like speed, acceleration, and deviation.

Using a leave-one-out method with Euclidean distance for matching, the system achieved a 46.67\% identification success rate with 30 samples from five users. Despite its limited accuracy, the prototype demonstrates the feasibility of mouse dynamics as a low-cost, non-intrusive authentication method for multifactor frameworks.

\subsection{Mouse Dynamics and Deep Neural Networks}

\parencite{20_Mouse_Coordinate} This study investigates deep learning models for static user authentication using mouse dynamics, comparing models like 1D-CNN, 2D-CNN, LSTM, and hybrid CNN-LSTM against a baseline SVM model. This study emphasizes transfer learning for small datasets, layered feature interpretation using Layer-Wise Relevance Propagation (LRP), and optimized preprocessing methods.

The 2D-CNN model demonstrated superior performance, achieving the highest AUC and lowest EER across datasets. Specifically, on the Balabit dataset, the 2D-CNN achieved an AUC of 0.964 and an EER of 4.2\%, significantly outperforming other architectures. The TWOS dataset results mirrored this trend, with the 2D-CNN showing strong adaptability and minimal performance variance. Key preprocessing, including normalization and sequence-length adjustments, further enhanced model robustness.

This study validates the efficacy of deep learning in mouse dynamics-based authentication and highlights challenges such as screen resolution sensitivity and data variability, suggesting areas for future improvement.

\subsection{Typing Style Using Text Length}

\parencite{30_Typing_Textlength} This study proposes a robust continuous user authentication (CUA) system based on keystroke dynamics, integrating a Recurrent Confidence Model (R-RCM) and ensemble learning. The approach authenticates users on each keystroke action using a dual-threshold system to detect imposters and manage genuine user deviations.

Experimental validation on the University of Buffalo dataset (75 users) demonstrated the system's effectiveness. Under the best-performing configuration (weighted classifier fusion with personalized RCM parameters), the system achieved a Mean Average Number of Imposter Actions (ANIA) of 9\% and a Mean Average Number of Genuine Actions (ANGA) of 80\%, highlighting rapid imposter detection and minimal disruptions to legitimate users. These results outperform traditional block-based authentication methods, ensuring heightened security with reduced user lockouts.

This research underscores the viability of keystroke dynamics for real-time, action-by-action authentication in high-security environments.

\subsection{Typing Style Using Keystrokes Data with PUC}
\label{subsec:typing-style-using-keystrokes-data}

\parencite{22_Keystroke} This study aims to identify a person based on the person's typing behavior using keystrokes data.
This study used the data from 46 students with at least 500 keystrokes each.
The paper found that using Pairwise User Coupling (PUC) with bottom-up tree structure based scheme works the best for identification.
It also observed that there is a higher accuracy when in the case of typing with both hands (89.7\%) compared to typing with one hand.
Also the paper shown that the duration of features are more stable than a combination of latency and duration features.

\subsection{Continuous Behavioral Biometric Authentication}
\label{subsec:continuous-behavioral-biometric-authentication}

\parencite{4_Continuous_Authentication}
This study presents a continuous authentication system based on behavioral biometrics, leveraging keystroke dynamics and mouse movements.

\subsection{Phone Usage}

\parencite{17_Phone} This study introduces Safeguard, a smartphone user reauthentication system utilizing sliding dynamics and pressure intensity as behavioral biometrics. By analyzing unique sliding patterns and on-screen pressure metrics, Safeguard employs Support Vector Machines (SVM) to achieve high accuracy with minimal system overhead. The system operates transparently, requiring only 15-20 sliding gestures to verify users with a False Acceptance Rate (FAR) of 0.03\% and a False Rejection Rate (FRR) of 0.05\%.

Data from 60 users were collected under various scenarios, including different apps and motion statuses, demonstrating the system's robustness and adaptability. Safeguard outperforms existing methods in accuracy and resistance to adversarial imitation. It is a scalable, hardware-independent solution suitable for real-time user reauthentication in diverse smartphone environments.
